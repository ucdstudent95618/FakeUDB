% Problem 1b goes below

The relations are the following: \\

Course(\underline{CID}, \underline{TERM}, SUBJ, CRSE, SEC, UNITS) \\
Meeting(\underline{CID}, \underline{TERM}, INSTRUCTOR(S), \underline{TYPE}, DAYS, TIME, BUILD, ROOM) \\
Enrolled(\underline{CID}, \underline{TERM}, \underline{SID}, SEAT, LEVEL, UNITS, CLASS, MAJOR, GRADE, STATUS) \\
Student(\underline{SID}, SURNAME, PREFNAME, EMAIL) \\

In the schema we will have four tables: Course, Student, Meeting, and Enrolled. \\

Course: \\
We have the following attributes: CID, TERM, SUBJ, CRSE, SEC, UNITS. CID, TERM, CRSE, and SEC are INT data types. SUBJ and UNITS are VARCHAR data types. The key is $\{CID, TERM\}$. \\

Meeting:\\
We have the following attributes: CID, TERM, INSTRUCTOR(S), TYPE, DAYS, TIME, BUILD, and ROOM. CID and TERM are INT data types. The remaining attributes are stored as VARCHAR. The key is $\{CID, TERM, TYPE\}$. \\

Enrolled: \\
We have the following attributes: CID, TERM, SID, SEAT, LEVEL, UNITS, CLASS, MAJOR, GRADE, and STATUS. CID, TERM, SID, and SEAT are of type INT. UNITS is of type DECIMAL. The rest are of type VARCHAR. The key is $\{CID, TERM, SID\}$. \\

Student: \\
We have the following attributes: SID, SURNAME, PREFNAME, and EMAIL. SID is stored as an INT data type. SURNAME, PREFNAME, and EMAIL are stored as VARCHAR data types. The key is $\{SID\}$. \\